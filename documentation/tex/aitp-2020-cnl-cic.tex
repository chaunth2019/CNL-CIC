\documentclass{easychair}


% PACKAGES
\usepackage{url}
\usepackage{amsmath}
\usepackage{amsthm}
\usepackage{amssymb}

% for underscores https://texfaq.org/FAQ-underscore
\usepackage{lmodern}
\usepackage[T1]{fontenc}
\usepackage{textcomp}
\usepackage{lineno}
\usepackage{outlines}

%\usepackage[
%bookmarksopen,
%bookmarksdepth=2,
%%breaklinks=true
%colorlinks=true,
%urlcolor=blue]{hyperref}

% GLOBAL FORMATTING
%\linenumbers
\parindent=0pt
\parskip=0.5\baselineskip
\raggedbottom

% TITLE AUTHOR DATE
\title{A Controlled Natural Language for Type Theory%
  \thanks{I thank
  Peter Koepke for introducing me to the field and Jesse Han for his
  contributions.  Research is supported by Sloan Foundation grant
  G-2018-10067.  This work is part of the Formal Abstracts project,
  which aims to capture all the major definitions and theorems of
  mathematics in a format that is both human and computer friendly.
  Source code and examples are found at
  \url{github.com/formalabstracts/CNL-CIC}.}  }


\author{Thomas Hales}

\institute{
University of Pittsburgh\\
Pittsburgh, PA, USA
}

%\date{March 5, 2020}

\authorrunning{T.Hales}
\titlerunning{Controlled Natural Language}



% THEOREMS 
\newtheorem{definition}{Definition}
\newtheorem{theorem}[definition]{Theorem}
\newtheorem{lemma}[definition]{Lemma}
\newtheorem{specification}[definition]{Specification}

% COMMANDS
\renewcommand{\iff}{\leftrightarrow}
\newcommand{\Prop}{\text{\tt Prop}}
\newcommand{\Type}{\text{\tt Type}}
\newcommand{\fld}{\textasciicircum}
\newcommand{\dequiv}{\mathrel{:=}} %{\mathrel{:\equiv}}
\newcommand{\Nat}{\ensuremath{{\mathbb N}}}
\newcommand{\Real}{\ensuremath{{\mathbb R}}}
\newcommand{\df}[1]{\text{\bf #1}}
\newcommand{\h}[1]{\text{#1}}
\newcommand{\join}{\lor}
\newcommand{\Mid}{\mathrel{\|}}
\newcommand{\comment}[1]{\%- \nobreak{#1}}
\renewcommand{\~}{\ }
\newcommand{\ignore}[1]{}
%\newcommand{\remark}[1]{(#1)}
\renewcommand{\_}{\textunderscore}
\renewcommand\labelitemi{-}
\renewcommand{\qed}{\ensuremath{\square}}

% ENVIRONMENTS

% \leavevmode\par is to make remark work when it is the first item in a subsection.
\newenvironment{remark}
{\leavevmode\par\begin{tabular}{|p{13cm}}\parskip=\baselineskip{\bf Remark.}}
{\end{tabular}}

\newenvironment{oblongo}{}{}

\newenvironment{prule}%
               {\begin{itemize}}%
               {\end{itemize}}
\newcommand{\ptem}{\item}
\newcommand{\nt}[1]{{\tt #1}}
\newcommand{\rw}{$\quad\to\quad$}



% DOCUMENT

\begin{document}
\maketitle

\section{Introduction}

This abstract describes the design and development of a controlled
natural language for mathematics that has the Lean theorem prover as
intended target.  We call this language Colada (short for
\emph{Co}ntrolled \emph{la}nguage \emph{da}ta).  Documents in our
dialect are written in a specially prepared \LaTeX\ file.  Our aim is
to capture definitions and theorem statements from the published
mathematical literature in our dialect, but checking mathematical
proofs is beyond the scope of our project.

Our design grows out of previous controlled natural languages for
mathematics (specifically, Forthel-Naproche-SAD), as described in
Peter Koepke's AITP 2019 talk, which exhibited some short proofs
written in fluent English that can be read and checked by their
software \cite{frerix2019making}, \cite{paskevich2007syntax},
\cite{lyaletski2004theorem}.

We use Forthel as the generic name for any of the dialects inspired by
Forthel (a controlled natural language developed by various
researchers starting with Glushkov's \emph{Evidence Algorithm}, and
implemented in Paskevich's PhD thesis), including Colada.  We refer to
the Colada language as \emph{our dialect}.  Our dialect differs from
others in that our semantic target is the logical Calculus of
Inductive Constructions (CiC) as implemented in the Lean theorem
prover, instead of first-order logic \cite{de2015lean}.  Our dialect
can be viewed as a fusion of three different syntactic traditions:
Forthel syntax, \LaTeX\ syntax, and Lean theorem-prover syntax.  From
another perspective, our dialect might be viewed as a mountain of
syntactic sugar for Lean.

\section{Controlled Natural Languages (CNL)}\label{sub:CNL}

By controlled natural language for mathematics (CNL), we mean an
artificial language for the communication of mathematics that is (1)
designed in a deliberate and explicit way with precise
computer-readable syntax and semantics, (2) based on a single natural
language (which for us is English), and (3) broadly understood at
least in an intuitive way by mathematically literate speakers of the
natural language.

CNLs can achieve a much higher degree of English fluency than other
proof-checking languages.  

Our basic aim is to develop a technology that lies somewhere between
the current practice of research mathematicians and the current
practice within the proof assistant community.

\section{Research to Date}

Our specific research contributions to date are as follows.

We have a design and specification of a controlled natural language.
Like other Forthel dialects, our grammar is not context-free.
However, it is similar to a context-free grammar by being specified
through production rules on terminal and nonterminal symbols.  Users
may extend the grammar with new mathematical notation and constructs:
the language contains syntax for the extension of its own syntax.

The lexical structure of our dialect is specified in sedlex, a lexical
generator tool for OCaml.  Our dialect has been specified in menhir,
an OCaml-based parser-generator tool for LR(1) grammars.  (Although
our dialect is not an LR(1) grammar, which prevents menhir from
automatically generating a parser, the software checks that our
grammar is well-formed.)

We believe that some complexity is justified (and even required) to
capture widespread mathematical idioms and formulas, the syntax of
type theory, and their interactions. Our grammar is recursive to an
extraordinary degree.  The grammar has about $350$ nonterminals and
about $550$ production rules.  The grammar contains about $150$
English words (such as \emph{all, any, are, case, define, exists, if,
  iff, is, no, not, of, or, over, proof, the, theorem}, etc.)  with a
fixed grammatical function.  User syntax extensions build on that
base.

We keep most features of Forthel, such as its handling of synonyms,
noun phrases, verbs, and adjectives; and its grammar extension
mechanisms.  We have added many additional features such as plural
formation for nouns and verbs, operator precedence parsing (with
user-specified precedence levels and associativities); scoping of
variables; syntax for \LaTeX\ macros; and dependent type theory
including inductive and mutual inductive types, structures, and lambda
terms.

A parser for our grammar has been implemented in OCaml, building
substantially on the parser combinator library that John Harrison
wrote to parse HOL Light.

Future work will transform parsed output to type-checked terms in
Lean: our system will output Lean-pre-expressions, which will then be
processed by Lean's elaboration and type-checking procedures.  Another
future project is syntax highlighting and auto-completion tools for
our dialect in editors such as emacs and VSCode. We also plan to
develop large mathematical libraries in our dialect.

We have written software that takes a specially prepared \LaTeX\ file
as input and strips away the non-semantic content (such as headers,
spaces and other layout, graphics, remarks, and dollar signs) and
outputs raw CNL.  The key to beautifully typeset \TeX\ documents is a
dual expansion system for macros.  The \TeX\ engine expands macros in
the usual way, but the CNL engine expands macros according to an
independent semantic specification.

We believe our language will find novel applications to search,
document analysis, and document transformation.





\newpage
\section{Example}

Examples will be given during the AITP presentation to show that English
fluency is obtained without loss of semantic content. The presentation
will include a discussion of dependent types, structures, inductive types,
type coercions, and implicit arguments.

Here is one example that
assumes a context in which a binary relation $(R,\le)$ has been defined.



\subsection{pdf} Here is a sample text, as viewed by the mathematician
reading the document.  

%begin sample
\def\deflabel#1{\begin{definition}[#1]\label{#1}}
\deflabel{greatest element} We say that $y$ is a
\df{greatest\~element} in $R$ iff for all\ $x,\ x \le y$.
\end{definition}

Let $x < y$ stand for $x \le y$ and $x \ne y$.
%end sample

\subsection{source}
The \LaTeX\ source file for the pdf is similar to documents
prepared every day by mathematicians.

\begin{verbatim}
\def\deflabel#1{\begin{definition}[#1]\label{#1}}

\deflabel{greatest element} We say that $y$ is a
\df{greatest\~element} in $R$ iff for all\ $x,\ x \le y$.
\end{definition}

Let $x < y$ stand for $x \le y$ and $x \ne y$.
\end{verbatim}

\subsection{CNL}  The CNL is generated from the source file by stripping formatting. 

\begin{verbatim}
Definition Label_greatest_element . We say that y is a 
greatestelement in R iff for all x , x \le y . 
 
Let x < y stand for x \le y and x \ne y . 
\end{verbatim}

\subsection{parse tree} This parse tree is generated from the CNL.  We only
display the first definition and have pruned the tree for simplicity.
Nonterminals are typeset in smallcaps and literals are in teletype.  At
hierarchical levels above those shown in the outline, we have
{\sc text\_item} $\to$ {\sc declaration} $\to$ {\sc definition}.  A further transformation not shown would produce
a Lean pre-expression from the tree.

\newcommand{\page}[1]{\rightskip=5pt \dotfill\quad%\rlap
{\hbox to 125pt{#1\hfill}}\par}

\begin{outline}
  \1 {\sc definition\_preamble}
  \2 {\sc lit\_def} \page{\tt Definition}
  \2 {\sc label} \page{\tt Label\_greatest\_element}
  \2 {\sc period} \page{\tt .}
  \1 list({\sc assumption})\page{}
  \1 {\sc definition\_affirm}
  \2 {\sc definition\_statement} $\to$ {\sc predicate\_def}
  \3 {\sc opt\_say} \page{\tt We say that}
  \3 {\sc predicate\_head} $\to$ {\sc predicate\_word\_pattern} $\to$ {\sc notion\_pattern}
  \4 {\sc tvar} \page {$y$}
  \4 {\sc lit\_is} \page{\tt is}
  \4 {\sc lit\_a} \page{\tt a}
  \4 {\sc word\_pattern} \page{\tt greatestelement in $R$}
  \3 {\sc iff\_junction} \page{\tt iff}
  \3 {\sc statement} \page{\tt for all $x$,\ $x\ \hbox{\tt \textbackslash le}\ y$}
  \2 {\sc period} \page{\tt .}
  \end{outline}




\bibliography{refs} 
\bibliographystyle{alpha}

\end{document}
