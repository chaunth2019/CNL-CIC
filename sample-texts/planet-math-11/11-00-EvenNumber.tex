\documentclass[12pt]{article}
\usepackage{pmmeta}
\pmcanonicalname{EvenNumber}
\pmcreated{2013-03-22 13:56:29}
\pmmodified{2019-11-14}
\pmowner{mathcam}{2727}
\pmmodifier{thales}{2727}
\pmformalizer{miinguyen}{}  % Note this new field
\pmtitle{even number}
\pmrecord{10}{34703}
\pmprivacy{1}
\pmauthor{mathcam}{2727}
\pmtype{Definition}
\pmcomment{trigger rebuild}
\pmclassification{msc}{11-00}
\pmclassification{msc}{03-00}
\pmrelated{NumberOdd}
\pmdefines{odd number}
\pmdefines{even integer}
\pmdefines{odd integer}
\pmdefines{even}
\pmdefines{odd}

\endmetadata

% this is the default PlanetMath preamble.  as your knowledge
% of TeX increases, you will probably want to edit this, but
% it should be fine as is for beginners.

% almost certainly you want these
\usepackage{amssymb}
\usepackage{amsmath}
\usepackage{amsfonts}

\usepackage{cnl}
\usepackage{xcolor}



% there are many more packages, add them here as you need them

% define commands here
\newcommand{\intpow}[2]{{#1}^{#2}}

\begin{document}

\parskip=\baselineskip

\begin{cnl}
\Cnlinput{../TeX2CNL/package/cnlinit}

\bigskip


In this section, let $k$, $r$ be integers.

 
\dfn{We say that $k$ is \df{odd} iff there exists an integer $r$ such
  that $k=2\* r+1$.}  
\dfn{We say that $k$ is \df{even} iff there
  exists an integer $r$ such that $k=2\* r$.}


\begin{demark}
The concept of even and odd numbers are most easily understood in the
binary base. Then the above definition simply states that even numbers
end with a $0$, and odd numbers end with a $1$.
\end{demark}

\lsubsubsection{Properties}


\begin{enumeratetext}
\item \thm{Every integer is either even or odd.} \inmark{This can be
  proven using induction, or using the fundamental theorem of
  arithmetic.}
\item \thm{Each integer $k$ is even if and only if $\intpow{k}{2}$ is
  even.}
\end{enumeratetext}


\end{cnl}
%%%%%
%%%%%
\end{document}
